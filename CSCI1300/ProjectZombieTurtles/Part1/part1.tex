\documentclass[12pt]{article}
 \usepackage[margin=1in]{geometry} 
\usepackage{amsmath,amsthm,amssymb,amsfonts}
 
\usepackage{listings}
\usepackage{color}
\usepackage{fontspec}

\newfontfamily\listingsfont[Scale=0.7]{Courier} 
\newfontfamily\listingsfontinline[Scale=.9]{Courier} 
\usepackage{color}
\definecolor{sh_comment}{rgb}{0.12, 0.38, 0.18 } %adjusted, in Eclipse: {0.25, 0.42, 0.30 } = #3F6A4D
\definecolor{sh_keyword}{rgb}{0.37, 0.08, 0.25}  % #5F1441
\definecolor{sh_string}{rgb}{0.06, 0.10, 0.98} % #101AF9
\def\lstsmallmath{\leavevmode\ifmmode \scriptstyle \else  \fi}
\def\lstsmallmathend{\leavevmode\ifmmode  \else  \fi}
\lstset {
 frame=shadowbox,
 rulesepcolor=\color{black},
 showspaces=false,showtabs=false,tabsize=2,
 numberstyle=\tiny,numbers=left,
 basicstyle= \listingsfont,
 stringstyle=\color{sh_string},
 keywordstyle = \color{sh_keyword}\bfseries,
 commentstyle=\color{sh_comment}\itshape,
 captionpos=b,
 xleftmargin=0.7cm, xrightmargin=0.5cm,
 lineskip=-0.3em,
 escapebegin={\lstsmallmath}, escapeend={\lstsmallmathend}
}
\newcommand{\lstJava}[1]{\lstinline[language=Java,breaklines=true,basicstyle= \listingsfontinline]$#1$}
\newcommand{\N}{\mathbb{N}}
\newcommand{\Z}{\mathbb{Z}}
 
\newenvironment{question}[2][Question]{\begin{trivlist}
\item[\hskip \labelsep {\bfseries #1}\hskip \labelsep {\bfseries #2.}]}{\end{trivlist}}
%If you want to title your bold things something different just make another thing exactly like this but replace "question" with the name of the thing you want, like theorem or lemma or whatever
 
\begin{document}
 
%\renewcommand{\qedsymbol}{\filledbox}
%Good resources for looking up how to do stuff:
%Binary operators: http://www.access2science.com/latex/Binary.html
%General help: http://en.wikibooks.org/wiki/LaTeX/Mathematics
%Or just google stuff
 
\title{Part 1 Zombie Project Writeup}
\author{Daniel Helfrich}
\maketitle
 
\begin{question}{1}
	
	The line, \lstJava{Universe un = new Universe(zTurtles,z,600, 600);}
	Creates a new universe or environment for the zombies. The universe
	will have size 600 x 600 and contain \lstJava{zTurtles} turtles and
	\lstJava{z} of which will be zombies.	

\end{question}

\begin{question}{2}
	Each iteration of the for loop runs the simulation exactly one 
	timestep. During each time step \lstJava{un.moveZombies()} and 
	\lstJava{un.zombieAttack()} methods are executed and do basically
	what they say they will do. If you wanted to run to simulation 1000
	times, you could set \lstJava{N = 1000}.
	
\end{question}

\begin{question}{3}
	The Universe constructor first creates a w x h canvas. Then it initializes an array of turtles with 
	\lstJava{numTurtles} elements. The \lstJava{tLocations} serves as an index that keeps track of all the 
	locations in the universe. It is a w x h array of integers. If there is no turtle at a particular 
	location, the array is set to $-1$ at that point. Otherwise, the array is set to the index of the turtle
	located at that point.

	The second for loop sets the locations of all the turtles. For the first \lstJava{numZombies} of the 
	turtles, are set to be zombies. The rest of them are set to be normal turtles. Also in the for loop, the 
	\lstJava{tLocations} variable is updated to reflect the locations of all the turtles.

	
\end{question}

\begin{question}{4}
	The instance method \lstJava{moveZombies()} is a function for the Universe class. It's goal, obviously is
	to move the zombies. It moves them up to 8 spaces in either direction for both the x and y direction. The
	line \lstJava{newX = min + (int)(Math.random() * ((max - min)+1));} creates a random integer between 
	\lstJava{min} and \lstJava{max}. The next line:\\ 
	\lstJava{if ((oldX+newX < 0) || (oldX+newX >= width))
		newX = 0;}
	Checks to see if adding this number the the location will move the turtle offscreen. If it will, then 
	the turtle will not move at all in that direction. The rest of the code for that method updates the
       	turtles' location as well as the \lstJava{tLocations} variable. (The old location is set to empty
	and the new location is set to the turtle's index.
	
\end{question}

\begin{question}{5}
	The for loop in \lstJava{moveZombies} iterates through every zombie and moves them as described before.
	
\end{question}

\begin{question}{6}
	In \lstJava{zombieAttack}, there are three nested for loops. The outermost for loop along with the next 
	if statement iterates through every infected zombie. The next four \lstJava{if / else} statements 
	construct a box with a given radius around the infected zombie that remains inside the grid. The next 
	two for loops iterate over every point in that radius and infects every turtle in the box.

\end{question}

\begin{question}{7}
	The radius variable is used to determine the size of the box that the zombies affect. This determines 
	how many points the for loops from the previous question loop over.
	
\end{question}

\begin{question}{8}
	The turtles are infected with the \lstJava{makeZombie()} method. This is applied to the turtles in the 
	radius. This updates the instance variable \lstJava{zombie} in the Turtle class. It turns from false to 
	true.
	
\end{question}

\begin{question}{9}
	The purpose of \lstJava{turtle[tLocations[x2][y2]]} is to refer to the instance of turtle and apply the 
	\lstJava{makeZombie()} method to it. This line lies after the if statement that checks whether there is 
	a turtle at that location.
	
\end{question}
\begin{question}{10}
	The visual display is updated in the \lstJava{setLocation()} method of the Turtles class. The method
	first erases the old turtle's location. Then it updates the new location in red if the turtle is a zombie
	and in blue otherwise.
	
\end{question}

\end{document}
              
            
